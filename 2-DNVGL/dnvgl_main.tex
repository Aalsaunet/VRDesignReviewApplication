% Carsten: 
% This chapter should come much earlier.
% Perhaps not everything, but at least DNV GL should be introduced and their motivations should be stated before even starting with the state-of-the-art reviews. 
% That means that section 5.1 of chapter 5 should be an own chapter between chapters 1 and 2. It would provide the fundamental reasons for making your thesis.
% In that new chapter, you should describe the approval workflow as it is today: papers is sent back and forth, models are made and updated on both sides, 
% annotations are disconnected from designs. And the vision: a share digital design where designer and reviewer can interact, perhaps even meet in a cooperative mode, 
% but without loosing the accountability that comes from today’s paper trail. This story of DNV GL’s vision is important - you must come back to it at the start of chapter 6.
% 
% It should also preferably be longer: more about what DNV GL is, what classification comprises, focus onto ships in this thesis. Where in the long line of 
% classification-related tasks does your thesis fit?
% The rest of 5 is in the right place. However, the current chapter 4 should really come after you have stated your requirement into this chapter.


\section{DNV GL and their motivations}
DNV GL is the world's largest classification society with more than 13 000 vessels and mobile offshore units, which represents a global market share of 21\%~\citep{TO:DNVGL}. 
It is the world's largest technical consultancy to onshore and offshore wind, wave, tidal, and solar industries, as well as the global oil \& gas industry 
-- 65\% of the world’s offshore pipelines are designed and installed to DNV GL technical standards~\citep{MTN:DNVGL}. 
A major part of DNV GL's work is evaluation and quality assurance of a client's product (e.g.~a ship) , 
where a DNV GL "Approval Engineer" conducts a design review of the client's model of the proposed product. 
This process usually consists of the following steps: 

\begin{enumerate}
	\item The designer sends the model to DNV GL for evaluation.
	\item The approval engineer inspects the model noting down aspects that doesn't meet DNV GL requirements.
	\item The designer receives the remarks and makes the necessary changes to the model.
	\item This process is repeated until both parties are satisfied.
\end{enumerate}

DNV GL is looking into the possibilities of digitilizing this process, and making it more interactive and efficient by using
virtual reality technology to conduct virtual design review meetings in the 3D models. 
As the sense of scale is important in a 3D model review, virtual reality technology is deemed promising as it gives a unique sense of scale
and a depth, which is hard to match by regular "2D screens". DNV GL is also interested in alternate interaction methods, as mouse and keyboard 
can have some limitation when working in a 3D environment~\citep{Rautaray2015}. 

% INCLUDE USE CASES HERE?
