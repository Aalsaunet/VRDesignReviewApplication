\documentclass[UKenglish]{ifimaster}  
\usepackage[utf8]{inputenc}           
\usepackage[T1]{fontenc,url}
\urlstyle{sf}
\usepackage{babel,textcomp,csquotes,duomasterforside,varioref,graphicx}
\usepackage[round]{natbib}
\usepackage{import}

\title{Using Virtual Reality- and Gesture Recognition Technology in Design Reviews}
\subtitle{A Master's Thesis}        
\author{Andreas Oven Aalsaunet}                  

\begin{document}
\duoforside[dept={Department of Informatics}, program={Programming and Networks}, long]                                        

\import{../0-preface/}{preface.tex}

\frontmatter{}                 
\tableofcontents{}
\listoffigures{}
\listoftables{}

\mainmatter{}
% \part{The state of the field}  
\chapter{Introduction}  
\import{../1-introduction/}{introduction.tex}

% About the field of gesture recognition, its challenges and its devices
%\chapter{Virtual Reality- and Gesture Recognition Technology}
\chapter{Virtual Reality Technology}
\import{../2-foundations/}{1-virtual_reality.tex}

\chapter{Gesture Recognition Technology}
\import{../2-foundations/}{2-gesture_recognition.tex}
\import{../2-foundations/}{3-devices.tex}
\import{../2-foundations/}{4-leap_motion.tex}
% About Virtual reality, with Oculus Rift and HTC Vive

% About commercial game engines such as Unity

% \part{The project} 

% About the design decitions and the use cases
\chapter{Designing the virtual design review application}
\import{../3-design/}{1-background.tex}
\import{../3-design/}{2-application_functionality.tex}
\import{../3-design/}{3-gesture-scheme.tex}

% About how this was implemented
\chapter{The Unity Implementation}
\import{../4-implementation/}{1-technology_choices.tex}

% Evaluation of the final product, user feedback etc
\chapter{Evaluation of the implementation}
\import{../5-evaluation/}{1-instructions.tex}

% Summary, future work and conclusion
\chapter{Conclusion}
\import{../6-conclusion/}{3-conclusion.tex}

\backmatter{}
\bibliography{references}
\bibliographystyle{apalike}
\end{document}
