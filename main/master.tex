\documentclass[UKenglish]{ifimaster}  
\usepackage[utf8]{inputenc}           
\usepackage[T1]{fontenc,url}
\urlstyle{sf}
\usepackage{babel,textcomp,csquotes,duomasterforside,varioref,graphicx}
\usepackage[round]{natbib}
\usepackage{import}

\title{Using Virtual Reality- and Gesture Recognition Technology in Design Reviews}
\subtitle{A Master's Thesis}        
\author{Andreas Oven Aalsaunet}                  

\begin{document}
\duoforside[dept={Department of Informatics}, program={Programming and Networks}, long]                                        

\frontmatter{}                 
\tableofcontents{}
\listoffigures{}
\listoftables{}

\import{../0-preface/}{preface.tex}

\mainmatter{}
\part{The state of the field}  
\import{../1-introduction/}{introduction.tex}

% About the field of gesture recognition, its challenges and its devices
\import{../2-gesture_recognition_technology/}{1-gesture_recognition.tex}
\import{../2-gesture_recognition_technology/}{2-challenges_and_problems.tex}
\import{../2-gesture_recognition_technology/}{3-devices.tex}
\import{../2-gesture_recognition_technology/}{4-leap_motion.tex}

% About Virtual reality, with Oculus Rift and HTC Vive

% About commercial game engines such as Unity

\part{The project} 

% About the design decitions and the use cases
\import{../3-design/}{1-background.tex}
\import{../3-design/}{2-application_functionality.tex}
\import{../3-design/}{3-gesture-scheme.tex}

% About how this was implemented
\import{../4-implementation/}{1-technology_choices.tex}

% Evaluation of the final product, user feedback etc
\import{../5-evaluation/}{1-instructions.tex}

% Summary, future work and conclusion
\import{../6-conclusion/}{3-conclusion.tex}

\backmatter{}
\bibliography{../references}
\bibliographystyle{apalike}
\end{document}
