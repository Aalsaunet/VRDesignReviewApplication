\section{Background}            
The field of virtual reality (VR) technology has seen an exciting development in recent years, 
with the release of the first commercial virtual reality headsets, such as Oculus Rift CV1 and HTC Vive, taking place in 2016.

The application area for these virtual reality headset have exceeded the expectations of many, with virtual reality 
technology being present in domains ranging from entertainment to educational training\citep{VRS2016}. 
\citet{VRS2016} reports numerous domains where virtual reality is successful being used, including 
healthcare (e.g surgery), military, architecture/construction, art, fashion, entertainment (games, films etc), education, business, telecommunications, sports and rehabilitation.

Despite this early success, there are still a lot challenges associated with virtual reality technology. One of these challenges is related to human-computer interactions
and will be expanded upon later in this chapter. This chapter will first discuss the virtual reality field and how gesture recognition technology can be very relevant for it,
before defining the problem definition, limitations and outline for the rest of this thesis.
%Making good use of virtual reality comes with several challenges, which will be discussed later. 

%Although these devices' initial target market was the games- and entertainment industry, 
%the applications of these devices in other domains have already exceeded the expectations of many~\citep{VRS2016}. 
% Examples of such domains include the military, the educational system, the healthcare industry, the construction businesses and the telecommunications industry~\citep{VRS2016}. 


\section{Virtual Technology}
Virtual reality can be defined as a realistic and immersive simulation of a three-dimensional 360 degree environment, 
created using interactive software and hardware, and experienced or controlled by movement of the body \citep{VRS2016}.

One of the most common ways to experience a virtual reality is through virtual reality headsets, which are stereoscopic head-mounted displays (HMD) 
that provide separate images for each eye \citep{POLYGON2016}. 
In addition to separate eye displays a HMD typically also contains head motion sensors such as gyroscopes, accelerometers 
and other sensors to track the user's head movements\citep{TW2016}. 
A person using a virtual reality head-mounted display should thus perceive a virtual world with realistic depth vision and be able to "look around" by turning his or her head.

The development of virtual reality head-mounted displays was in many ways fueled by the development of smart phones as many of the components are similar (e.g.~gyroscopes), and
these components also became more affordable by the prominence of smart phones. 
% This led to the prototype HMD "Oculus Rift Development Kit 1", released by Oculus VR in 2012, 
% being the first independently developed and sold virtual reality headset\citep{TW2016}. 

\begin{figure}%[h!] %[H]
	\includegraphics[width=\linewidth]{pictures/oculus_rift_dk1.jpg}
	\caption[The Oculus Rift Development Kit 1]{The Oculus Rift Development Kit 1, released by Oculus VR in 2012.}
	\label{fig:oculus}
\end{figure}

As virtual reality technology enables users to experience virtual worlds in a new way, 
human-computer interaction (HCI) is also a highly relevant topic. 
This field has in many ways seen a resurgence as virtual technology gives new possibilities, but also set new constrains. 
One of these constraints is limiting the user's field of vision exclusively to that projected by the lenses, 
which may make interaction with traditional input devices, such as mouse and keyboard, more challenging. 
Because of this, alternate methods of interacting with the computer is a relevant topic. 
One of these methods is the use of gestures, 
which have long been considered an interaction technique that can potentially deliver more natural, 
creative and intuitive methods for communicating with our computers~\citep{Rautaray2015}. 
To enable the use of gestures as a viable input method to a computer, responsive and reliable gesture recognition techniques are needed.  

\section{Problem definition}
This thesis will evaluate the consequences of utilizing virtual reality technology in combination with vision based gesture recognition technology, and discuss the benefits it 
might bring, as well and the challenges it presents. 
The thesis will also review the design and implementation of a design review application, which is developed as part of this thesis with the aforementioned goal in mind. 
The design review application is also a prototype developed for the major international classification company DNV GL to evaluate how the use of virtual reality and gesture
recognition technology might benefit their design review process. As such, the application requirements has been created in cooperation with DNV GL and represents
common 3D object manipulating and navigation tasks. After discussing the design and implementation choices of this application, the user evaluation session will be discussed. 
The user evaluation sessions were performed in cooperation with DNV GL employees, the potential end users, and contained invaluable feedback relevant to the use of virtual reality and gesture
recognition technology in a professional setting. 

% Write more about the concrete thesis ideas and why all the chapters are relevant. 


\section{DNV GL}
Intro to DNV GL. Who and what are they?

\section{Limitations}
The initial list of application features had to be shortened significantly to focus more on the most relevant parts for this thesis. As such the design review application
is more a prototype or proof-of-concept than a finished product. Section 5.2.1 outlines the application features and will explain more of whats include in the application and what isn't. 

\section{Outline}
This thesis is organized as follows: In chapter 2 we will review the history and theoretical foundations for virtual reality, as well as discuss its performance demands
and the various challenges it presents. In chapter 3 we will discuss gesture recognition technology, its concepts and the different technology enabling this technology.
Chapter 4 will review the Leap Motion Controller, a vision-based gesture recognition device, both in terms of its hardware and software properties. Especial
emphasis will be put on its API (application program interface), which significantly simplifies the creation of programs utilizing gesture recognition.
Chapter 5 we will review the design of the design review application, its user stories (i.e function requirements) and what gestures it makes use of.
Chapter 6 will go more into the technical details of how the application is implemented in the Unity game engine and review its architecture along with some central Unity concepts.
In chapter 7 the user evaluation sessions will be covered, and the responses discussed and analyzed.
Chapter 8 will conclude this thesis with a summary of the findings and some thoughts about future work.