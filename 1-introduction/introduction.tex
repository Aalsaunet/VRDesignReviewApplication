\chapter{Introduction}                  
The field of virtual- and augmented reality technologies has seen an exciting development in recent years, with the first release of commercial virtual reality devices, such as Oculus Rift and HTC Vive, taking place in 2016. These devices enable the user to be completely emerged in 3D virtual worlds by having one high resolution lens per eye, covering the user's whole field of vision. Although these devices' initial target market was the games- and entertainment industry, the applications of these devices in other domains have already exceeded the expectations of many~\citep{VRS2016}. Examples of such domains include the military, the educational system, the healthcare industry, the construction businesses and the telecommunications industry~\citep{VRS2016}. 

As virtual reality technology enables users to visually perceive virtual worlds in a new way, human-computer interaction (HCI) is also a highly relevant topic. This field has in many ways seen a resurgence as virtual technology gives new possibilities, but also set new constrains. One of these constraints is limiting the user's field of vision exclusively to that projected by the lenses, which may make interaction with traditional input devices, such as mouse and keyboard, more challenging. Because of this, alternate methods of interacting with the computer is a relevant topic. One of these methods is the use of gestures, which have long been considered an interaction technique that can potentially deliver more natural, creative and intuitive methods for communicating with our computers~\citep{Rautaray2015}. 

\section{Outline}
The upcoming master's thesis will explore the possibilities of utilizing virtual reality- and gesture recognition technology in combination during the design review process of the major international classification company DNV GL. This will be accomplished by implementing a design review application that makes use of both these technologies for several design review tasks (see section 2.1), and by having users evaluate the effectiveness of these techniques. This essay will summarize DNV GL's design review process, list the application requirements, discuss some of the developments within the field of gesture recognition technology and briefly review the Leap Motion Controller.
