\section{Background}           
The field of virtual reality technology has seen an exciting development in recent years, 
with the release of the first commercially successful virtual reality headsets, such as the Oculus Rift CV1 and HTC Vive, taking place in 2016.
The application areas for these virtual reality headsets have exceeded the expectations of many, with virtual reality 
technology already present in several domains, ranging from engineering to entertainment~\citep{VRS2016}. 
\citet{VRS2016}, among others, reports numerous domains where virtual reality is successfully used, including 
healthcare (e.g in surgery), military, architecture/construction, art, fashion, entertainment (games, films etc), education, business, telecommunications, sports and rehabilitation.

With the success of virtual reality technology, and increased attention towards technologies that complement it, several businesses and institutions 
are interested in making use of the new possibilities virtual reality offer~\citep{TW22016}. One major business area for virtual reality (besides entertainment)
has been in the architecture, engineering and construction fields. Iris VR, a New York-based technology company building virtual reality applications, has reported that
among their 15 000 customers, 75\% are from these industry segments. One possible reason for the impact virtual reality has had on these fields is their dependence 
on big and complex 3D models. 

As virtual reality head mounted devices (HMD) are stereoscopic, i.e provide separate images for each eye, they are able to deliver a feeling of 
depth and scale that is unrivaled by regular two dimensional displays~\citep{POLYGON2016}. This point was highlighted in an interview of a senior designer at an architect 
firm, conducted by~\citet{TW22016}. He stated that "Practically nobody can understand architectural drawings, and even 3D visualizations are a stretch for most. 
But everybody gets VR instinctively. You can get to the point very quickly. It either sells or kills the project right away." 

Virtual reality applications that allow its users to inspect models in 3D also have many additional possibilities. 
One such possibility could be to annotate (alternatively edit or comment) on the model while being virtually "inside" it 
(i.e when using the application and wearing a virtual reality HMD) and 
to use the virtual reality application as a design and collaboration tool to exchange ideas about the model.

DNV GL, the world's largest maritime classification society, is looking into exactly this, and view this as a potential big improvement over their current 
"paper-based" work flow. More specifically, DNV GL is interested in a virtual reality application for design reviews, a classification process where DNV GL employees review 
clients' design models and comment on various aspects of the models that need to be improved to meet the classification requirements. 

This thesis addresses this vision and utilizes several state-of-the-art frameworks and technologies to design and implement such 
an application. This presents several challenges, some of which are mentioned in the next section and reviewed more throughout the thesis, while 
DNV GL, their workflow, visions and motivations is discussed further in chapter~\ref{chapter:dnvgl}. 


\section{The Challenges of Virtual Reality} 
Despite the early success the field of virtual reality technology has seen, there are still a lot challenges associated with it. 
These challenges include prevention of virtual reality sickness (a kind of induced motion sickness), 
strict performance demands on target hardware and having more suitable input methods when using virtual reality HMDs. % (human-computer interactions).
Addressing these challenges, in both design and implementation, is an important step when building virtual reality software~\citep{OCULUS2016}, 
and each of these challenges, among others, will be discussed more in the following chapters.

As virtual reality technology enables users to experience virtual worlds in a new way, 
human-computer interaction (HCI) is also a highly relevant topic. 
This field has in many ways seen a resurgence as virtual technology gives new possibilities, but also set new constraints. 
One of these constraints is limiting the user's field of vision exclusively to that projected by the lenses, 
which may make interaction with traditional input devices, such as mouse and keyboard, more challenging. 
Because of this, alternate methods of interacting with the computer is a relevant topic. 
One of these methods is the use of gestures, 
which have long been considered an interaction technique that can potentially deliver more natural, 
creative and intuitive methods for interacting with computers~\citep{Rautaray2015}. 
To enable the use of gestures as a viable input method to a computer, responsive and reliable gesture recognition techniques are needed.  

\section{Problem Definition}
This thesis reviews how the current state-of-the-art virtual reality- and vision-based gesture recognition technologies can be utilized in a professional capacity, 
and more specifically, to the design reviews of complex 3D models. 
Instead of printing the model to paper, drawing on the paper, scanning it and sending it by email - as isn't an uncommon workflow today, 
the designer and reviewer could have a virtual design review meeting, were they could meet in the 3D model, interact, survey- and annotate it together. 
They could manipulate the model, and change how the annotations appear in the models.
These annotations could be stored in a database - a system keeping track of their history, information and states - and be accessed from multiple platforms - like 
an issue tracker or a digital scrum board. The interaction, surveying, and experience as a whole, could be enhanced by virtual reality technology -
giving the unique sense of scale and a depth that is invaluable for a design - 
and gesture recognition technology - allowing the users to work with the 3D model in new and innovative ways. 

This is the vision of DNV GL, who regularly conducts such design reviews. % Hmm, ta bort
To address this, and evaluate this vision's feasibility, this thesis implements such a design review application, and 
reviews its requirements and design, documents its implementation details and evaluates its performance by conducting user test sessions.
In the prototype application, developed as part of this thesis, the user is able to navigate 3D models, annotate them - by either creating 
annotation orbs, which are physical object in the model, or by attaching an annotation directly to an object - and perform various other actions. 
This can be done in a conventional manner, i.e by mouse, keyboard and a display, by only using gestures (hand movements) and a virtual reality headset, or 
a combination of the two.

To ensure that the application is developed using state-of-the-art technology, and that it addresses the challenges such technology presents, 
the field of virtual reality is reviewed and discussed. In addition to this, the gesture recognition field is also reviewed, as this alternate 
input method can have the potential to increase the usability of 3D-based applications - such as the design review application -, especially when coupled with virtual reality.
Several software frameworks and devices that enable this technology are also discussed in this thesis, as we wanted to quickly prototype the 
application using these state-of-the-art tools.
We chose to use readily available virtual reality tools, such as the virtual reality headsets Oculus Rift CV1 and HTC Vive, coupled with
the Leap Motion Controller, a stereoscopic vision-based gesture recognition device. The software was developed using the Unity 
framework - a popular game engine - because of its capability and as it allowed us to interface with other projects in DNV GL


% This vision was pursued by implementing an application that enabled the user to navigate the model and create annotation - 
% i.e comments or remarks to an aspect of the model - that were position in the model and either represented as "annotation orbs" or as an object highlighting.
% The design review application is also a prototype developed for the major international classification company DNV GL to evaluate how the use of virtual reality and gesture
% recognition technology might benefit their design review process. As such, the application requirements have been created in cooperation with DNV GL and represents
% common 3D object manipulating and navigation tasks


% This thesis will evaluate the consequences of utilizing virtual reality technology in combination with vision based gesture recognition technology, and discuss the benefits it 
% might bring, as well and the challenges it presents. 
% The thesis will also review the design and implementation of a design review application, which is developed as part of this thesis with the aforementioned goal in mind. 
% The design review application is also a prototype developed for the major international classification company DNV GL to evaluate how the use of virtual reality and gesture
% recognition technology might benefit their design review process. As such, the application requirements have been created in cooperation with DNV GL and represents
% common 3D object manipulating and navigation tasks. After discussing the design and implementation choices of this application, the user evaluation session will be discussed. 
% The user evaluation sessions were performed in cooperation with DNV GL employees, the potential end users, and contained invaluable feedback relevant to the use of virtual 
% reality and gesture
% recognition technology in a professional setting. 
% We chose to use readily available VR tools, Oculus Rift and Leap Motion for this review, and based the software development 
% of Unity because it allowed us to interface with other projects in DNV GL.

% Write more about the concrete thesis ideas and why all the chapters are relevant. 

\section{Scope and Limitations}
The initial list of application features had to be shortened significantly to focus more on the most relevant parts for this thesis. As such the design review application
is more a prototype or proof-of-concept than a finished product. 
Section~\vref{sec:application_functionality} outlines the application features and explains more of what's included in the application and what isn't. 

\section{Research Methods}
The ACM Task Force report \textit{Computing As a Discipline}, by~\citet{Denning1989}, identifies a structure of how research in computing should be approached.
This report defines computer science as an intersection between several processes, with the primary being applied mathematics, science and engineering~\citep{Denning1989}.
These central processes are basically reflected in the paradigms of \textit{theory}, \textit{abstraction} and \textit{design}.

The first paradigm, \textit{theory}, is defined by (i) characterizing the objects of study (definition), (ii) hypothesizing possible relationships among them (theorem), (iii) 
determining whether the relationships are true (proof) (iv) and interpreting the results.

The second paradigm, \textit{abstraction}, is defined by (i) forming a hypothesis, (ii) constructing a model and making a prediction, (iii) designing an experiment and collecting data, 
and (iv) analysing the results. 

The third paradigm, \textit{design}, is defined by (i) stating requirements, (ii) stating specifications, (iii) designing and implementing the system and (iv) test the system.

% which can briefly be summarized as:
% \begin{itemize}
%     \item (i) characterize objects of study (definition), hypothesize possible relationships among them (theorem), determines whether the relationships are true (proof)
%             and interpret the results.
%     \item (ii) form a hypothesis, construct a model and make a prediction, design an experiment and collect data, and analyse results 
% \end{itemize}

In this thesis the \textit{design paradigm} is followed, as the main focus of this thesis is to design and implementation an application using the technology of interest,
and test it by user evaluations. The initial requirements are discussed section~\vref{sec:initial_design}, before being scoped in section~\vref{sec:application_functionality}. 
The specifications and design are discussed in chapter~\vref{chapter:design}, while the implementation is reviewed and documented in chapter~\vref{chapter:implementation}. 
The testing and evaluation of the system is covered in chapter~\vref{chapter:evaluation}

\section{Main Contributions}
The summarized main contributions of this thesis are:
\begin{enumerate}
    \item A discussion of classification societies, and more specifically DNV GL, and how the use of virtual reality and gesture recognition technology could
            benefit them in their design review process.
    \item A review of the state-of-the-art virtual reality technology, with special focus on the head-mounted devices Oculus Rift CV1 and HTC Vive, 
            in addition to its challenges and techniques for addressing this.
    \item A survey on what's known about virtual reality- and simulator sickness, what their main causes are, and which of these causes are primarily due to (i) 
            individual differences in susceptibility or (ii) the application design or performance.
    \item A review of the state-of-the-art gesture recognition technology, its history and what the main techniques are. After a general overview 
            we will focus more on vision-based gesture recognition technology, and more specifically stereoscopic vision technology, 
            as it is arguably the most promising gesture recognition technology. 
    \item Design and implementation of a prototype application utilizing virtual reality- and gesture recognition technology to perform the basic functionality
          required of a design review application (e.g~navigating and annotating).
    \item Evaluating several aspect of the design review implementation through user testing, a discussions of what design aspects worked better than other
            in such an application and how mature the virtual reality- and gesture recognition technologies are.
    \item A conclusion which sums up important finding, formulates new hypothesis and recommends subject for further study.
\end{enumerate}


\section{Outline}
This thesis is organized as follows: In chapter~\ref{chapter:dnvgl} DNV GL and their business domains and processes are introduced, together with a general discussion regarding
the role of classification societies. In this chapter we also define some of the scope for the application, a topic which is revisited in 
chapter~\ref{chapter:design}. In chapter~\ref{chapter:vr} we review the history, concepts and demands of virtual reality, as well as discuss the issue of virtual reality
sickness and other challenges. The implication these challenges have for the design and implementation stages, which are covered in chapter~\ref{chapter:design} and
chapter~\ref{chapter:implementation}, are also discussed. Chapter~\ref{chapter:grt} reviews gesture recognition technology and its exciting possibilities for virtual reality.
This chapter also discusses the different technologies that makes up the field of gesture recognition technology, and how it functions. 
In chapter~\ref{chapter:design} we review the design of the design review application, its various use cases and function requirements, how gesture recognition technologies
can be used, and what frameworks and technologies are utilized. Chapter~\ref{chapter:technical} review the software development libraries, APIs and frameworks
which were outlined in chapter ~\ref{chapter:design} and are utilized in the implementation. Special emphasis is put on important concepts of the Unity Engine, its programming model
and on the Leap Motion library.  
In chapter~\ref{chapter:implementation} we document how the application is implemented and expand upon some of the Unity and Leap Motion concepts that were used.
In chapter~\ref{chapter:evaluation} the user evaluation sessions are covered, and the responses discussed and analyzed.
Chapter~\ref{chapter:conclusion} concludes this thesis with a summary and provides some ideas for future topics of research.


% This thesis is organized as follows: In chapter 2 we will review the history and theoretical foundations for virtual reality, as well as discuss its performance demands
% and the various challenges it presents. In chapter 3 we will discuss gesture recognition technology, its concepts and the different technology enabling this technology.
% Chapter 4 will review the Leap Motion Controller, a vision-based gesture recognition device, both in terms of its hardware and software properties. Especial
% emphasis will be put on its API (application program interface), which significantly simplifies the creation of programs utilizing gesture recognition.
% Chapter 5 we will review the design of the design review application, its user stories (i.e function requirements) and what gestures it makes use of.
% Chapter 6 will go more into the technical details of how the application is implemented in the Unity game engine and review its architecture along with some central Unity concepts.
% In chapter 7 the user evaluation sessions will be covered, and the responses discussed and analyzed.
% Chapter 8 will conclude this thesis with a summary of the findings and some thoughts about future work.