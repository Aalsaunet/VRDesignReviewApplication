\section{Background}           
The field of virtual reality technology has seen an exciting development in recent years, 
with the release of the first commercially successful virtual reality headsets, such as the Oculus Rift CV1 and HTC Vive, taking place in 2016.
The application area for these virtual reality headset have exceeded the expectations of many, with virtual reality 
technology already being present in several different domains, ranging from engineering to entertainment~\citep{VRS2016}. 
\citet{VRS2016} reports numerous domains where virtual reality is successful being used, including 
healthcare (e.g surgery), military, architecture/construction, art, fashion, entertainment (games, films etc), education, business, telecommunications, sports and rehabilitation.

With the success of virtual reality technology, and increased attention towards technologies that complement it, several businesses and institutions 
are interested in making use of the new possibilities virtual reality offer~\citep{TW22016}. One major business area for virtual reality, besides entertainment, 
has been in the architecture, engineering and construction fields. Iris VR, a New York-based technology company building virtual reality applications, has reported that
among their 15 000 customers, 75\% are from these industry segments. One possible reason for the impact virtual reality has had on these fields is their dependence 
on big and complex 3D models. As virtual reality head mounted devices (HMD) are stereoscopic, i.e provide separate images for each eye, they are able to deliver a feeling of 
depth and scale that is unrivaled by regular two dimensional displays~\citep{POLYGON2016}. This point was highlighted in an interview of a senior designer at an architect 
firm, conducted by~\citet{TW22016}. He stated that "Practically nobody can understand architectural drawings, and even 3D visualizations are a stretch for most. 
But everybody gets VR instinctively. You can get to the point very quickly. It either sells or kills the project right away." 

Virtual reality applications that allow its users to inspect models in 3D also has many additional possibilities. 
One example is to be able to work (e.g.~to edit, annotate or comment) on the model while being "inside" it (e.g.~when wearing a virtual reality HMD) and 
to use the virtual reality application as a design and collaboration tool to exchange ideas about the model.
DNV GL, the world's largest maritime classification society, is looking into exactly this, and view this as a potential big improvement over their standard 
"paper-based" work flow. More specifically, DNV GL is interested in a virtual reality application for design reviews, a classification process where DNV GL employees review 
a client's design model and comments on various aspects of the model that need to be improved to meet the classification requirements. 

This thesis will address this vision and utilize several state-of-the-art frameworks and technologies to design and implement such 
an application. This presents several challenges, some of which will be mentioned in the next section and reviewed more throughout the thesis, while 
DNV GL, their work flow, visions and motivations will be discussed further in chapter~\ref{chapter:dnvgl}. 


\section{The Challenges of Virtual Reality} 
Despite the early success the field of virtual reality technology has seen, there are still a lot challenges associated with it. 
These challenges include prevention of virtual reality sickness (a kind of induced motion sickness), 
stricter performance demands on target hardware and having more suitable input methods when using virtual reality HMDs. % (human-computer interactions).
Addressing these challenges, in both design and implementation, is an important step when building virtual reality software~\citep{OCULUS2016}, 
and each of these challenges, among others, will be discussed more in the following chapters.

As virtual reality technology enables users to experience virtual worlds in a new way, 
human-computer interaction (HCI) is also a highly relevant topic. 
This field has in many ways seen a resurgence as virtual technology gives new possibilities, but also set new constrains. 
One of these constraints is limiting the user's field of vision exclusively to that projected by the lenses, 
which may make interaction with traditional input devices, such as mouse and keyboard, more challenging. 
Because of this, alternate methods of interacting with the computer is a relevant topic. 
One of these methods is the use of gestures, 
which have long been considered an interaction technique that can potentially deliver more natural, 
creative and intuitive methods for interacting with computers~\citep{Rautaray2015}. 
To enable the use of gestures as a viable input method to a computer, responsive and reliable gesture recognition techniques are needed.  

\section{Problem definition}
This thesis will evaluate the consequences of utilizing virtual reality technology in combination with vision based gesture recognition technology, and discuss the benefits it 
might bring, as well and the challenges it presents. 
The thesis will also review the design and implementation of a design review application, which is developed as part of this thesis with the aforementioned goal in mind. 
The design review application is also a prototype developed for the major international classification company DNV GL to evaluate how the use of virtual reality and gesture
recognition technology might benefit their design review process. As such, the application requirements has been created in cooperation with DNV GL and represents
common 3D object manipulating and navigation tasks. After discussing the design and implementation choices of this application, the user evaluation session will be discussed. 
The user evaluation sessions were performed in cooperation with DNV GL employees, the potential end users, and contained invaluable feedback relevant to the use of virtual reality and gesture
recognition technology in a professional setting. 

% Write more about the concrete thesis ideas and why all the chapters are relevant. 

\section{Limitations}
The initial list of application features had to be shortened significantly to focus more on the most relevant parts for this thesis. As such the design review application
is more a prototype or proof-of-concept than a finished product. Section 5.2.1 outlines the application features and will explain more of whats include in the application and what isn't. 

\section{Outline}
This thesis is organized as follows: In chapter~\ref{chapter:dnvgl} DNV GL and their business domains and processes will be introduced, together with a general discussion regarding
the role of classification societies. In this chapter we also define some of the scope for the application, a topic which will be revisited in 
chapter~\ref{chapter:design}. In chapter~\ref{chapter:vr} we will review the history, concepts and demands of virtual reality, as well as discuss the issue of virtual reality
sickness and other challenges. The implication these challenges have for the design and implementation stages, which are covered in chapter~\ref{chapter:design} and
chapter~\ref{chapter:implementation}, are also discussed. Chapter~\ref{chapter:grt} reviews gesture recognition technology and its exciting possibilities for virtual reality.
This chapter also discusses the different technologies that makes up the field of gesture recognition technology, and how it functions. 
In chapter~\ref{chapter:design} we will review the design of the design review application, its user stories (i.e function requirements), how gesture recognition technologies
can be used, and what frameworks and technologies are utilized. Chapter~\ref{chapter:technical} will review the software development libraries, APIs and frameworks
which were outlined in ~\ref{chapter:design} and is utilized in the implementation. Special emphasis will be put important concepts of the Unity Engine, its programming model
and on the Leap Motion library.  
In chapter~\ref{chapter:implementation} we will document how the application is implemented and expand upon some of the Unity and Leap Motion concepts that were used.
In chapter~\ref{chapter:evaluation} the user evaluation sessions will be covered, and the responses discussed and analyzed.
Chapter~\ref{chapter:conclusion} will conclude this thesis with a summary of the findings and some thoughts about future work.


% This thesis is organized as follows: In chapter 2 we will review the history and theoretical foundations for virtual reality, as well as discuss its performance demands
% and the various challenges it presents. In chapter 3 we will discuss gesture recognition technology, its concepts and the different technology enabling this technology.
% Chapter 4 will review the Leap Motion Controller, a vision-based gesture recognition device, both in terms of its hardware and software properties. Especial
% emphasis will be put on its API (application program interface), which significantly simplifies the creation of programs utilizing gesture recognition.
% Chapter 5 we will review the design of the design review application, its user stories (i.e function requirements) and what gestures it makes use of.
% Chapter 6 will go more into the technical details of how the application is implemented in the Unity game engine and review its architecture along with some central Unity concepts.
% In chapter 7 the user evaluation sessions will be covered, and the responses discussed and analyzed.
% Chapter 8 will conclude this thesis with a summary of the findings and some thoughts about future work.