\chapter{Gesture recognition devices}
Creation and implementation of such efficient and
accurate hand gesture recognition systems is aided by two major types of enabling
technologies for human-computer interaction, namely contact-based and vision-based devices~\citep{Rautaray2015}. 

\paragraph{Contact-based devices} are usually wearable objects, such as gloves or armbands, which register the user's kinetic movement through sensors and attempt to mirror it in the virtual world. Some notable products making use of this technology include the Nintendo Wii remote controller and the Myo armband (see figure \ref{fig:myo}). 

\begin{figure}%[h!] %[H]
	\includegraphics[width=\linewidth]{pictures/myo_armband.jpg}
	\caption{The Myo armband is a gesture recognition device worn on the forearm and manufactured by Thalmic Labs. The Myo enables the user to control technology wirelessly using various hand motions. It uses a set of electromyographic (EMG) sensors that sense electrical activity in the forearm muscles, combined with a gyroscope, accelerometer and magnetometer to recognize gestures~\citep{Myo2015}.}
	\label{fig:myo}
\end{figure}

\paragraph{Vision-based devices} usually make use of either depth-aware cameras or stereo cameras to approximate a 3D representation of what's output by the cameras, which in many ways are similar to how the human eyes work. Products making use of this technology include the Microsoft's Kinect and the Leap Motion controller (see figure \ref{fig:leapmotion}). 

\begin{figure}%[h!] %[H]
	\includegraphics[width=\linewidth]{pictures/leapmotion2.png}
	\caption{The Leap Motion controller is a small USB peripheral device which is designed to be placed on a physical desktop, facing upward. Using two monochromatic IR cameras and three infrared LEDs, the device observes a roughly hemispherical area, to a distance of about 1 meter, and generates almost 200 frames per second of reflected data~\citep{LeapMotion2016}.}
	\label{fig:leapmotion}
\end{figure} 

\paragraph{}Both approaches have their advantages and disadvantages (see~\citet{Rautaray2015} for a deeper discussion of these). Contact-based devices generally have a higher accuracy of recognition and a lower complexity of implementation than that of vision-based ones. Vision-based devices are on the other hand seen as more user friendly as they require no physical contact with the user. 

The main disadvantage of contact-based devices is the potential health hazards, which may be caused by some of its components~\citep{Schultz2003}. Research has suggested that mechanical sensor materials may raise symptoms of allergy and magnetic component may raise the risk of cancer~\citep{Nishikawa2003}. Even though vision-based devices have the initial challenge of complex configuration and implementations, they are still considered more user friendly and hence more suited for usage in long run. Because of the reasons outlined above the master's thesis will primarily be oriented towards vision-based gesture recognition technologies. 

\section{The primary Vision-based Technologies}
Today, there are three primary vision-based technologies that can acquire 3D images: Stereoscopic vision, structured light pattern and time of flight (TOF)~\citep{Ko2012}.
These all make use of one or several cameras and lights to capture and recognize certain movements or poses from the user, and transform it to a certain action on the computer (e.g.~a recognized finger tap might be the equivalent to left mouse button click). 

\paragraph{Stereoscopic vision}is the most common 3D acquisition method and uses two cameras to obtain a left and right stereo image. These images are slightly offset on the same axis as the human eyes. As the computer compares the two images, it develops a disparity image that relates the displacement of objects in the images.

\paragraph{Structured light pattern}measure or scan 3D objects through illumination. Light patterns are created using either a projection of lasers or LED light interference or a series of projected images. By
replacing one of the sensors of a stereoscopic vision system with a light source, structured-light-based technology basically exploits the same triangulation as a stereoscopic system does to acquire the 3D coordinates of the object. Single 2D camera systems with an IR- or RGB-based sensor can be used to measure the displacement of any single stripe of visible or IR light, and then the coordinates can be obtained through software analysis.

\paragraph{Time of flight}is a relatively new technique among depth information systems
and is a type of light detection and ranging (LIDAR) system that transmits a light pulse from an emitter to an object. A receiver determines the distance of the measured object by calculating the travel time of the light pulse from the emitter to the object and back to the receiver in a pixel format.

\begin{figure}%[h!] %[H]
	\includegraphics[width=\linewidth]{pictures/Vision-based_comparisons.png}
	\caption{Comparison of Vision-based sensor technologies~\citep{Ko2012}.}
	\label{fig:VBComparisions}
\end{figure} 

\paragraph{Of these technologies} stereoscopic vision is perhaps the most promising one for the consumer market as it has the lowest material cost~\citep{Ko2012}, and has proved more reliable in variable light conditions than its counterparts. One of the latest consumer-oriented devices of this kind is the Leap Motion Controller, which distinguishes itself for having a higher localization precision than other depth vision-based devices~\citep{Weichert2013}, and also for capturing depth data related to palm direction, fingertips positions, palm center position, and other relevant points~\citep{Wei2016}. The Leap Motion Controller will be reviewed more in-depth in the next section. 

\section{How vision-based devices functions}