\section{DNV GL and their motivations}
DNV GL is the world's largest classification society with more than 13 000 vessels and mobile offshore units, which represents a global market share of 21\%~\citep{TO:DNVGL}. It is the world's largest technical consultancy to onshore and offshore wind, wave, tidal, and solar industries, as well as the global oil \& gas industry -- 65\% of the world’s offshore pipelines are designed and installed to DNV GL technical standards~\citep{MTN:DNVGL}. A major part of DNV GL's work is evaluation and quality assurance of a client's product (e.g.~a ship) , where a DNV GL "Approval Engineer" conducts a design review of the client's model of the proposed product. This process usually consists of the following steps: 

\begin{enumerate}
	\item The designer sends the model to DNV GL for evaluation.
	\item The approval engineer inspects the model noting down aspects that doesn't meet DNV GL requirements.
	\item The designer receives the remarks and makes the necessary changes to the model.
	\item This process is repeated until both parties are satisfied.
\end{enumerate}

DNV GL is looking into the possibilities of making this process more interactive, effective and simultaneous using virtual reality- and gesture technologies to conduct virtual design review meetings in the 3D models. The idea is to create sessions that enable both parties to be virtually present in the same instance of the 3D model, and to interact with it. The parties present should be able to navigate the 3D model, annotate problems, link the problems to predefined rules or requirements and see representations of each other (i.e through a character model/avatar). The application should also support a model versioning system to keep track of which model version(s) annotations are tied to.

At the end of a session the 3D model may contain one or several annotations, which the designer then has to address before the design can be approved. A more detailed list of application demands is described below.
