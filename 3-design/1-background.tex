\section{DNV GL and their motivations}
DNV GL is the world's largest classification society with more than 13 000 vessels and mobile offshore units, which represents a global market share of 21\%~\citep{TO:DNVGL}. 
It is the world's largest technical consultancy to onshore and offshore wind, wave, tidal, and solar industries, as well as the global oil \& gas industry 
-- 65\% of the world’s offshore pipelines are designed and installed to DNV GL technical standards~\citep{MTN:DNVGL}. 
A major part of DNV GL's work is evaluation and quality assurance of a client's product (e.g.~a ship) , 
where a DNV GL "Approval Engineer" conducts a design review of the client's model of the proposed product. 
This process usually consists of the following steps: 

\begin{enumerate}
	\item The designer sends the model to DNV GL for evaluation.
	\item The approval engineer inspects the model noting down aspects that doesn't meet DNV GL requirements.
	\item The designer receives the remarks and makes the necessary changes to the model.
	\item This process is repeated until both parties are satisfied.
\end{enumerate}

DNV GL is looking into the possibilities of digitilizing this process, and making it more interactive and efficient by using
virtual reality technology to conduct virtual design review meetings in the 3D models. 
As the sense of scale is important in a 3D model review, virtual reality technology is deemed promising as it gives a unique sense of scale
and a depth, which is hard to match by regular "2D screens". DNV GL is also interested in alternate interaction methods, as mouse and keyboard 
can have some limitation when working in a 3D environment~\citep{Rautaray2015}. As mentioned in the previous chapters this thesis will use 
the Leap Motion Controller, a vision-based device using stereoscopic cameras, as a primary input device to the application. 

\section{Initial design ideas} 
The core functionality in the application should be to navigate the 3D model and "annotate" it (i.e creating and placing remarks tied to a the model), 
primarily by using the advantages of virtual reality and gesture recognition. 
The users should in later iteration also be able to create "sessions" that enable several users to be virtually present 
in the same instance of the 3D model, and to interact with it using gestures. During a session a user should then be able to create annotations, 
which can be interacted with (e.g.~edited or deleted) and is tied to the 3D model and the session. 
Beyond this there is a lot of other functionality which should be in place for a complete product, but which will not be a priority 
for this thesis as the virtual reality and gesture recognition aspects are the focus. 

In the final product the application should support a lot more functionality, some of which is described in the next section.

%To enable the application to a be a VR collaboration tool

\subsection{Application use cases}
This section gives an overview of the use cases which is intended for the finished application, some of which will be implemented in this thesis. 
This section separates the user stores into two subsection "The Launcher" and "The Inspector", where "The Inspector stories" was all implemented, while
"The Launcher stories" where skipped because they aren't relevant for the thesis, and to give more time to prioritize "The Inspector stories".


\import{}{2-application_functionality.tex}

% link the problems to predefined rules or requirements 

% see representations of each other (i.e through a character model/avatar).