% Carsten: Once more, the reasons for reading this chapter must be stated clearly at the
% start of this chapter, and it must be founded on section 1.3, or perhaps be a
% consequence of knowledge found in chapter 2, which leads to the assumption that gestures are a desirable approach to interaction. 
% If it is a consequence, than the motivation for looking deeper into gestures should be included in the conclusion of chapter 2.

\section{Gesture recognition devices}
Gesture recognition technology is a field that has gained much attention with the growth of the virtual reality field, 
and it's a very diverse one with roots in sensor technology, image processing and computer vision~\citep{Vafadar2014}. 
The first attempts at a commercial hand gesture recognition system were typically glove-based control interfaces, often called \textbf{\textit{data gloves}} 
and were gloves with sensors attached to it. As the image processing and computer vision technology wasn't mature yet, these \textbf{\textit{contact-based devices}} remained 
the primary gesture recognition technology, until the image processing-reliant \textbf{\textit{vision-based devices}} began to see some success in the 2000s~\citep{Premaratne2014}.
Another factor which made data gloves ideal was a very limited requirement for processing power, as any pre-processing were rarely done, 
and thus the systems could run optimally on the commodity 1980s and 1990s computers~\citep{Premaratne2014}.  

\begin{figure}%[h!] %[H]
	\includegraphics[width=\linewidth]{pictures/old_dataglove.png}
	\caption[The Z Glove]{The Z Glove, developed by Zimmerman in 1982. Picture from \citet{Premaratne2014}}
	\label{fig:old_dataglove}
\end{figure} 

Today, both contact-based and vision-based devices are utilized for gesture recognition purposes. 

\paragraph{Contact-based devices} are usually wearable objects, such as gloves or armbands, 
which register the user's kinetic movement through sensors and attempt to mirror it in the virtual world. 
Some notable products making use of this technology include the Nintendo Wii remote controller and the Myo armband (see figure~\vref{fig:myo}). 

\begin{figure}%[h!] %[H]
	\includegraphics[width=\linewidth]{pictures/myo_armband.jpg}
	\caption[The Myo armband]{The Myo armband is a gesture recognition device worn on the forearm and manufactured by Thalmic Labs. 
	The Myo enables the user to control technology wirelessly using various hand motions. 
	It uses a set of electromyographic (EMG) sensors that sense electrical activity in the forearm muscles, combined with a gyroscope, 
	accelerometer and magnetometer to recognize gestures~\citep{Myo2015}.}
	\label{fig:myo}
\end{figure}

\paragraph{Vision-based devices} usually make use of either depth-aware cameras or stereo cameras to approximate a 3D representation of what's output by the cameras, 
which in many ways are similar to how the human eyes work. 
Products making use of this technology include the Microsoft's Kinect and the Leap Motion controller (see figure~\vref{fig:leapmotion}). 

\begin{figure}%[h!] %[H]
	\includegraphics[width=\linewidth]{pictures/leapmotion2.png}
	\caption[The Leap Motion Controller]{The Leap Motion Controller is a small USB peripheral device which is designed to be placed on a physical desktop, 
	facing upward. Using two monochromatic IR cameras and three infrared LEDs, the device observes a roughly hemispherical area, to a distance of about 1 meter, 
	and generates almost 200 frames per second of reflected data~\citep{LeapMotion2016}.}
	\label{fig:leapmotion}
\end{figure} 

\paragraph{}Both approaches have their advantages and disadvantages (see~\citet{Rautaray2015} for a deeper discussion of these). 
Contact-based devices generally have a higher accuracy of recognition and a lower complexity of implementation than vision-based ones. 
Vision-based devices are on the other hand seen as more user friendly as they require no physical contact with the user. 

The main disadvantage of contact-based devices is the potential health hazards, which may be caused by some of its components~\citep{Schultz2003}. 
Research has suggested that mechanical sensor materials may raise symptoms of allergy and magnetic component may raise the risk of cancer~\citep{Nishikawa2003}. 
Even though vision-based devices have the initial challenge of complex configuration and implementations, 
they are still considered more user friendly and hence more suited for usage in long run. Because of the reasons outlined above this thesis will primarily 
be oriented towards vision-based gesture recognition technologies. 

\subsection{The primary Vision-based Technologies}
Today, there are three primary vision-based technologies that can acquire 3D images: Stereoscopic vision, structured light pattern and time of flight (TOF)~\citep{Ko2012}.
These all make use of one or several cameras and lights to capture and recognize certain movements or poses from the user, 
and transform it to a certain action on the computer (e.g.~a recognized finger tap might be the equivalent to left mouse button click). 

\paragraph{Stereoscopic vision}is the most common 3D acquisition method and uses two cameras to obtain a left and right stereo image. 
These images are slightly offset on the same axis as the human eyes. As the computer compares the two images, 
it develops a disparity image that relates the displacement of objects in the images.

\paragraph{Structured light}measure or scan 3D objects through illumination. Light patterns are created using either a projection of lasers or LED light 
interference or a series of projected images. 
By replacing one of the sensors of a stereoscopic vision system with a light source, structured-light-based technology basically exploits the same triangulation as a 
stereoscopic system does to acquire the 3D coordinates of the object. 
Single 2D camera systems with an IR- or RGB-based sensor can be used to measure the displacement of any single stripe of visible or IR light, 
and then the coordinates can be obtained through software analysis.

\paragraph{Time of flight}is a relatively new technique among depth information systems
and is a type of light detection and ranging (LIDAR) system that transmits a light pulse from an emitter to an object. 
A receiver determines the distance of the measured object by calculating the travel time of the light pulse from the emitter to the object and back to the receiver 
in a pixel format.

\begin{figure}%[h!] %[H]
	\includegraphics[width=\linewidth]{pictures/Vision-based_comparisons.png}
	\caption{Comparison of Vision-based sensor technologies~\citep{Ko2012}.}
	\label{fig:VBComparisions}
\end{figure} 

\paragraph{Of these technologies} stereoscopic vision is perhaps the most promising one for the consumer market as it has the lowest material cost~\citep{Ko2012}, 
and has proved more reliable in variable light conditions than its counterparts. 
One of the latest consumer-oriented devices of this kind is the Leap Motion Controller, 
which distinguishes itself for having a higher localization precision than other depth vision-based devices~\citep{Weichert2013}, 
and also for capturing depth data related to palm direction, fingertips positions, palm center position, and other relevant points~\citep{Wei2016}. 
The Leap Motion Controller will be reviewed more in-depth in the next chapter. 

%\subsection{How vision-based devices functions}

\section{Gesture Recognition Principles}
A gesture can be defined as a physical movement of the hands, arms, face and body with the intent to convey information or meaning~\citep{Mitra2007}, 
Even though the use of keyboard and mouse is a prominent interaction method, there are situations in which
these devices are impractical for human-computer interaction (HCI). This is particularly the case for interaction with 3D objects~\citep{Rautaray2015}. 

To be able to convey semantically meaningful commands through the use of gestures one must rely on a gesture recognition system, 
which is responsible for capturing and interpreting gestures from the user and, if applicable, carry out the desired action. 
Often this process is seen as a sum of three fundamental phases: Detection, tracking and recognition~\citep{Rautaray2015}.
This section will describe what makes up a gesture recogntion system, with special emphasis on hand gesture recognition, and
summarize some common challenges with vision-based gesture recognition methods.

\subsection{Static and dynamic gestures}
In the gesture recognition field it is common to define a gesture as either a static or dynamic. \textbf{\textit{Static gestures}} can in simple terms be defined as gestures
without any movement. The hand and its fingers and joints simply maintain a certain position or orientation and it is recognized as a gesture. One example 
of this gesture category is the "V sign" (or the "peace sign"), where the index and middle fingers are raised and parted while the other fingers are clenched.

\textbf{\textit{Dynamic gestures}}, on the other hand, are gesture that involves or requires movement for the gesture to have meaning. One example of this
might be to wave goodbye to someone or to twist a straight hand back and forth to indicate uncertainty. One can classify dynamic gestures into several 
subclasses, such as conscious gestures, which are done intentionally for communication purposes, or unconscious gestures, which are carried out unconsciously.
See~\vref{fig:static_and_dynamic} for an hierarchical overview.
%As the gestures included in this thesis' implementation can be considered static (they use movement, but does not require it)

\begin{figure}%[h!] %[H]
	\includegraphics[width=\linewidth]{pictures/static_and_dynamic.png}
	\caption[The vision-based hand gesture categories]{The vision-based hand gesture categories~\citep{Kanniche2009}.}
	\label{fig:static_and_dynamic}
\end{figure} 

\subsection{Detection}
The first step in a typical gesture recognition system is to detect the relevant parts of the captured image and segment them from the rest. 
This segmentation is crucial because it isolates the relevant parts of the image from the background to ensure that only the relevant part is processed by the subsequent 
tracking and recognition stages~\citep{Cote2006}. 
A gesture recognition system will typically be interested in hand gestures, head- and arm movements and body poses, and thus only these factors should be observed by the system.
A gesture recognition system interested in detecting e.g.~hand gestures should thus only consider hands as a relevant segment, and thus only observe these.

Many different detection methods have been proposed by research, each using different visual features to detect relevant segments. 
Example of such visual features include skin color, shape, motion and anatomical models of the hands~\citep{Cote2006}.

\paragraph{Color} detection is a method of detecting the relevant segment (e.g.~hands) by its color. 
When employing this method one important decision is what color space to use, though color spaces efficiently separating the chromaticity from
the luminance components of color are typically the preferred ones. These are favored as they have some degree of robustness to illumination variability, which
is a weakness of this detection method. In addition to this skin color detection also have performance problems when the background contains objects that have a 
color distribution similar to human skin, although this can be combated by \textit{background subtraction}, and with variability in human skin tones~\citep{Rautaray2015}.

\paragraph{Shape} detection is a method of detecting the relevant segment by its shape, and usually tries to extract the contours of objects to judge
whether those objects are relevant or not. An advantage with this method over color detection is that it's not directly dependent on skin color or
illumination, although these are still a factor~\citep{Rautaray2015}. However, a major disadvantage with this methods relates to occlusion and viewpoint
problems, which might cause a hand to not be recognized as one because of the camera angle and/or the hands orientation and configuration. One way to prevent this
might be to use several cameras with different viewpoints.
Shadows can also cause a problem as shadows of a hand often will be detected as hands themselves. Because of these disadvantages it is more
common to use this method in combination with other ones rather than on its own.

\paragraph{Motion} detection is a method of detecting the relevant segment though motion, and assumes that all moving object are relevant.
When used as a gesture recognition scheme it requires a very controlled setup as it assumes that the only motion in the image is caused by hand movement. 
This method is also more common to use in combination with other methods.

\begin{figure}%[h!] %[H]
	\includegraphics[width=\linewidth]{pictures/gr_pipeline.png}
	\caption[The gesture recognition pipeline]{A typical gesture recognition pipeline~\citep{Pisharady2015} }
	\label{fig:gr_pipeline}
\end{figure}

\subsection{Tracking}
The second step in a gesture recognition system is to track the movements of the relevant segments of the frames, e.g.~the hands. 
Tracking can be described as the frame-to-frame correspondence of the segmented hand regions and aims to understand the observed hand movements. 
This is often a difficult task as hands can move very fast and their appearance can change vastly within a few frames, 
especially when light condition is a big factor~\citep{Wang2010}. 
One additional note is that if the detection method used is fast enough to operate at image acquisition frame rate, it can also be used for tracking~\citep{Rautaray2015}.   


\subsection{Recognition}
The last step of a gesture recognition system is to detect when a gesture occurs. 
This often implies checking against a predefined set of gestures, each entailing a specific action. 
To detect static gestures (i.e postures involving no movement) a general classifier or template-matcher can be used, 
but with dynamic gestures (which involves movement) other methods, which keep the temporal aspect, such as a Hidden Markov Model (HMM), are often required~\citep{Benton1995}. 
The recognition technology often makes uses of several methods from the field of machine learning, including supervised, unsupervised and reinforced learning.

When a gesture recognition system detects a relevant segment, it is thus tracked and represented in some way in the system. For hand gesture representations, 
which is the most relevant for this thesis, there are two major categories of hand gesture representations: 3D model-based methods and appearance-based methods~\citep{Rautaray2015}.

\begin{figure}%[h!] %[H]
	\includegraphics[width=\linewidth]{pictures/gesture_representation.png}
	\caption[Vision-based hand gesture representations]{Vision-based hand gesture representations~\citep{Bourke2007} }
	\label{fig:gesture_representations}
\end{figure}

% Carsten: Conclude the chapter with insights gained for the thesis, and consequences for your work on the thesis, such as choices made and work you have to do. 
% In this particular case, the conclusion must also create a smooth transition to chapter 4, because it’s focussing more on a special case 
% of the things that you’ve just introduced.
% Provide already in the conclusion of 3 a hint why there is so much to tell about the Leap Motion that it’s worth its own chapter.