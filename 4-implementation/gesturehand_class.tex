The \texttt{GestureHand} class is assigned as a component to each hand object, and contain several important variables.
\begin{itemize}
	\item \texttt{bool isLeftHand} - Keeps track of whether this instance belong to the left or the right hand.
	\item \texttt{GameObject handModel} - A reference to the unity game-object that contains the handModel this gesturehand instance is relevant for.
	\item \texttt{Material[] handMaterials} - An array of different material for the hand model. These are swapped between when e.g.~a certain gesture is brecognized and tracked.
	\item \texttt{GestureHand otherHand} - A reference to the other GestureHand-instance. For the lefthand GestureHand-instance this variable thus point to the right hand GestureHand
			hand-instance.
	\item \texttt{GameObject detectors} - A reference to the gameobject that hold/is parent of this hands detectors.
	\item \texttt{bool combineGestures} - Keeps track of whether the user is using a combined XYZ axis gesture scheme or if these movement gesture are kept separate (see~\vref{sec:combined-movement}).
	\item \texttt{Vector gestureOriginPosition} - Holds the x-, y- and z- coordinates of the palm when the current gesture was recognized.
	\item \texttt{HandState handState} - Holds one of several HandState enum values that represent the hand state. This variable has one of the following enum values:
			NONE = 0, PINCH = 1, PALM\_DOWN = 2, PALM\_SIDE = 3, FIST = 4, SINGLE\_POINT = 5, DOUBLE\_POINT = 6 or DISABLED = 7 (see~\vref{tab:handstates}).  
\end{itemize}

\begin{table}[]
\centering
\label{tab:handstates}
\begin{tabular}{p{1cm}| p{3.5cm} | p{6cm}}
	\textbf{ID} &  \textbf{Variable name} &   \textbf{Implication} \\\\
	0 & NONE & No gesture is being performed. \\\\
	1 & PINCH & User can rotate by the y- and z-axis \\\\
	2 & PALM\_DOWN & User can move along the y-axis.  \\\\
	3 & PALM\_SIDE & User can move along the x-axis.  \\\\
	4 & FIST & User can move along the z-axis.  \\\\
	5 & SINGLE\_POINT & User is placing/has placed point-annotation.  \\
	6 & DOUBLE\_POINT & User is placing/has placed object-annotation.  \\
	7 & DISABLED & The detectors are disabled and no gesture and hand state can be achieved before enabling them.  
\end{tabular}
\caption{The \texttt{GestureHand} class' hand states}
\end{table}

The \texttt{GestureHand} class also contains several important functions that are called when a certain gesture is activated or deactivated.
\texttt{void ActivateGesture(int gestureCode)} is called by a detector when it becomes active, i.e when its criteria are met and the gesture it represent is recognized.
It then signals the \texttt{GestureHand} class with its code/signature so \texttt{GestureHand} knows which detectors called it.
If a pinch gesture is detected by the left hand pinch detector the left hand \texttt{GestureHand} class' \texttt{ActivateGesture} is thus called with the argument "1".
From a design standpoint one should be able to send the Handstate.PINCH enum as a value, but then this function wouldnt be exposed in the Unity Inspector interface (which
only seem to accept primitive or built-in argument types). 

Once this function is called it sets the hand state to the value of the gestureCode, sets the gesture origin position and switches the hand materials.
Hand material switches is done by assigning the material at index \texttt{gestureCode} in the \texttt{handMaterials} list to the hand model, so 
if a pinch gesture is a performed the hand model is assigned the material at \texttt{handMaterials[1]}. The HandState enums and the hand material list thus 
follow the same sorting order. 

The \texttt{GestureHand} class also has the function \texttt{void DeactivateGesture()}, which is called by a detector when it becomes deactivated. 
This function simply resets the hand state by setting it to HandState.NONE (NONE = 0) and assign the material at \texttt{handMaterials[(int) HandState.NONE]} (i.e 0) to the
handmodel. \texttt{GestureHand} also contains the functions \texttt{enableDetectors()} and \texttt{disableDetectors()}, which 
enables or disables all the detectors that belongs to the same hand as the current \texttt{GestureHand} instance. These are used in two scenarios:
When the user switches between having gestures enabled and disable by using the menu options "Enable Gestures" and "Disable Gestures" and when an annotation 
is edited. When an annotation is edited, and the annotation form is open, gestures that one can use otherwise (e.g~the movement and rotation gestures) are disabled.

\begin{figure}%[h!] %[H]
	\includegraphics[width=\linewidth]{pictures/screenshots/gestures/disabled_detector.png}
	\caption[Disabled gestures]{When gestures are disabled, i.e all the detectors are disabled, the hand models are transparent}
	\label{fig:disabled_detector}
\end{figure} 