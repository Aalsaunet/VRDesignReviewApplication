The master controller represents a collection of controllers, which all have the role of handling input from the user and translate it into
the correct action. These controllers typically interact with the \texttt{GestureHand} class to check if a certain hand state-criterion
is met, utilize its utility functions and check for changes every frame. The \texttt{GestureHand} will be covered more in depth in section~\vref{sec:gesturehand_class}, 
but can in short be described as a class which is instantiated once per hand and keeps track of what potential gesture the hands are performing, the origin coordinates
for the potential gestures and other useful hand-specific information. 
The following four sections will cover the controllers that the master controller contains.

\subsection{The Rotation Controller}
The \texttt{RotationController} is a script component of the \texttt{MasterController} and its primary function to handle user input related to rotation.
\texttt{RotationController} contains a number of instance variables, which will be described below. Some of these variables have a public access modifier, as
this allows their values to be seen and edited from the Unity Inspector view (see figure~\vref{fig:unity_inspector} for an example). Variables that does not
have this requirement, and is which should not be accessed from other parts of the application, are given a private access modifier.

\begin{itemize}
    \item \texttt{public GestureHand leftHand, rightHand} - Stores the \texttt{GestureHand} instances that represents the left and right hand.
    \item \texttt{public float sensitivity} - A float-point multiplier that determines the sensitivity of the rotational actions. 
                                              All statements that rotates the camera is multiplied by this variable's value, which by default is 100.0.
    \item \texttt{public float clampAngle} - An absolute value in degrees for the maximum rotation allowed along in y-axis. 
                                            Its default is 90.0 (degrees), meaning that the user can rotate from looking straight ahead (0.0 degrees on the y-axis) to 
                                            straight up (90.0 degrees on the y-axis) and straight down (-90.0 degrees on the y-axis). Note that this rotational 
                                            limitation along the y-axis is in place to prevent the user from rotating the cameras (and the enire \texttt{MasterController}) 
                                            "upside-down". %Include an illustration?                                         
    \item \texttt{private float rotX, rotY} - The floatpoint variables rotX and rotY are intermediate values that stores the rotation of the \texttt{MasterController} (and thus the
                                              cameras) and is used to calculate the new rotation quaternion that is applied to the \texttt{MasterController}'s transform 
                                              every frame. Note that \texttt{RotationController} only allows rotation
                                              along the x-axis (left-and-right) and y-axis (up-and-down), and not along the z-axis (a "barrel roll" rotation). 
    \item \texttt{private InteractionBox iBox} - The \texttt{InteractionBox} class is a Leap Motion abstraction for the area (i.e the "box") above/in front of 
                                                the Leap Motion device that is interactable (i.e where the device can detect and track the hands), 
                                                and is used for normalizing purposes. 
                                                This variable holds a reference to the latest InteractionBox-object, which is updated on every Leap Motion-frame by
                                                retrieving it from the Leap Motion \texttt{Frame} object (see table~\vref{table:annotation_visibility_code} for an example).  
\end{itemize}


\subsection{The Movement Controller}
The \texttt{MovementController} is a script component of the MasterController, and relates to the movement of the user.

\subsection{The Raycast Controller}

\subsection{The Annotation Form Controller}
