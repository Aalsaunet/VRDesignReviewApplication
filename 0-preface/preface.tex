\chapter*{Abstract}       
Classification societies date back to the second half of the 18th century, where marine insurers developed a system for independent technical assessment of the ships
presented to them for insurance cover. Today, a major part of a classification society's responsibilities is to review the designs of enormous maritime
vessels. This usually involves working with big and complex 3D models and 3D tools, but without support to do many of the tasks required in a design review.
As a consequence, the workflow is often just partially digital, and many important tasks, such as annotating or commentating on aspects of the models, are done on paper.

DNV GL, the world's largest maritime classification society, is interested in digitalizing this process more, and make it more interactive and efficient by 
utilizing an application that allows for \textit{virtual} design review meetings in the 3D models. In these virtual design review meetings, 
the designer and reviewer could remotely interact, survey the model together, and annotate it instead of model-printouts.
As the sense of scale is important in a 3D model review, virtual reality technology is deemed promising as it gives a unique sense of scale
and a depth, which is hard to match by regular 2D screens. 
DNV GL is also interested in alternate interaction methods, as mouse and keyboard can have some limitation when working in 3D environments. 
Gesture Recognition Technology has been of special interest as this can potentially offer unique approaches to working with 3D models.

This thesis implements such a design review application using state-of-the-art virtual reality- and vision-based gesture recognition technologies,
coupled with the Unity game engine, a popular cross-platform game development platform and software framework. 
After discussing these technologies' theoretical foundations, the thesis
reviews the requirements and design of the design review application, in addition to documenting its implementation and evaluating its performance by conducting user tests.
In the implemented design review application the user is able to navigate 3D models, annotate them and perform various other actions, all performed by gestures. 


% The field of virtual reality technology has seen an exciting development in recent years, 
% with the release of the first commercially successful virtual reality headsets, such as the Oculus Rift CV1 and HTC Vive, taking place in 2016.
% The application areas for these virtual reality headsets have exceeded the expectations of many, with virtual reality 
% technology already present in several domains, ranging from engineering to entertainment.
% 
% Despite this early success, there are still a lot challenges associated with virtual reality technology. This thesis will discuss several of these challenges, 
% with especial attention to human-computer interaction, and more specifically to vision-based gesture recognition as an input method used in combination with virtual reality technology. 
% This thesis is also interested in virtual reality's applicability to business and engineering, and will also review an associated implementation of a virtual reality-based 
% design review application made for the company DNV GL. In addition to discussing the design and implementation details of this implementation, the thesis will also summarize 
% findings made during user tests of the application. 

\chapter*{Acknowledgements}  
First of all, I would like to thank my four supervisors: Carsten Griwods and Pål Halvorsen from Simula Research Laboratory, and Ole Christian Astrup and Ovidiu Valentin Drugan 
from DNV GL. Each of them have contributed to the making of this thesis through feedback, suggestions, motivation and assistance, and I am grateful for your help.
Secondly, I want to thank my dear Rebecca Tholin, who has been patient with me - especially these last couple of days before the submission deadline - and also
been a great motivator and proof-reader throughout this process. I'm grateful that you are both such a lovely person and a pro at english grammar.

I also want to thank Zach Kinstner, for creating the Hover UI Kit and allowing me to use it in the thesis implementation,  
Jakub Michalski for giving a great crash course on Unity at the DNV GL offices, as well as all the test users from DNV GL, 
who provided invaluable feedback on the functionality of the thesis implementation as well as plenty of ideas for new ones.

Lastly, I want to thank by parents, Inger and Ole, and by brother Einar, for the support and encouragement. \\\\


\noindent
Andreas Oven Aalsaunet \\
Oslo, May 2, 2017





% Specially thanks to Carsten Griwodz of Simula Research Laboratory (where there actually is such a thing as a free lunch!).  

