\chapter*{Abstract}                    
% Traditionally the idea of several people interacting in a virtual world, and the emerging virtual reality technologies (e.g Oculus Rift and HTC Vive etc) have been 
% closely tied to the gaming- and entertainment industry. 
% As of this date, these technological advances remains mostly irrelevant for most other industry segments, but could this be about to change? 
% This essay will explore the possibilities of applying these technologies to the design and engineering segment of the industry.

The field of virtual reality (VR) technology has seen an exciting development in recent years, 
with the release of the first commercial virtual reality headsets, such as Oculus Rift CV1 and HTC Vive, taking place in 2016.
The application area for these virtual reality headset have exceeded the expectations of many, with virtual reality 
technology being present in domains ranging from entertainment to educational training.

Despite this early success, there are still a lot challenges associated with virtual reality technology. This thesis will discuss several of these challenges, 
with especial attention to human-computer interaction, and more specifically to vision-based gesture recognition as an input method used in combination with virtual reality technology. 
This thesis is also interested in virtual reality's applicability to business and engineering, and will also review an associated implementation of a virtual reality-based 
design review application made for the company DNV GL. In addition to discussing the design and implementation details of this implementation, the thesis will also summarize 
findings made during user tests of the application. 

\chapter*{Acknowledgements}  
% Specially thanks to Carsten Griwodz of Simula Research Laboratory (where there actually is such a thing as a free lunch!).  

