\chapter*{Abstract}                    
% Traditionally the idea of several people interacting in a virtual world, and the emerging virtual reality technologies (e.g Oculus Rift and HTC Vive etc) have been 
% closely tied to the gaming- and entertainment industry. 
% As of this date, these technological advances remains mostly irrelevant for most other industry segments, but could this be about to change? 
% This essay will explore the possibilities of applying these technologies to the design and engineering segment of the industry.

The field of virtual reality (VR) technology has seen an exciting development in recent years, 
with the release of the first commercial virtual reality headsets, such as Oculus Rift CV1 and HTC Vive, taking place in 2016.
The application area for these virtual reality headset have exceeded the expectations of many, with virtual reality 
technology being present in domains ranging from entertainment to educational training.

Despite this early success, there are still a lot challenges associated with virtual reality technology. This thesis will discuss several of these challenges, 
with especial attention to human-computer interaction, and more specifically to vision-based gesture recognition as an input method used in combination with virtual reality technology. 
This thesis is also interested in virtual reality's applicability to business and engineering, and will also review an associated implementation of a virtual reality-based 
design review application made for the company DNV GL. In addition to discussing the design and implementation details of this implementation, the thesis will also summarize 
findings made during user tests of the application. 

\chapter*{Acknowledgements}  
First of all, I would like to thank my four supervisors: Carsten Griwods and Pål Halvorsen from Simula Research Laboratory, and Ole Christian Astrup and Ovidiu Valentin Drugan 
from DNV GL. Each of them have contributed to the making of this thesis through feedback, suggestions, motivation and assistance, and I am grateful for your help.
Secondly, I want to thank Rebecca Tholin, who has been patient with me - especially these last couple of days before the submission deadline - and also
been a great motivator and proof-reader throughout this process. I'm grateful that you are both such a lovely person and a pro at english grammar.

I also want to thank Zach Kinstner, for creating the Hover UI Kit and allowing me to use it in the thesis implementation,  
Jakub Michalski for giving a great crash course to Unity at DNV GL offices, as well as all the test users from DNV GL, 
who provided invaluable feedback on the functionality of the thesis implementation as well as plenty of ideas for new ones.

Lastly, I want to thank by parents, Inger and Ole, and by brother Einar, for the support and encouragements. \\\\


\noindent
Andreas Oven Aalsaunet \\
Oslo, May 2, 2017





% Specially thanks to Carsten Griwodz of Simula Research Laboratory (where there actually is such a thing as a free lunch!).  

