\section{The core design} 
The core functionality in the application should be to navigate the 3D model and "annotate" it (i.e creating and placing remarks tied to a the model), 
primarily by using the advantages of virtual reality and gesture recognition. 
The users should in later iteration also be able to create "sessions" that enable several users to be virtually present 
in the same instance of the 3D model, and to interact with it using gestures. During these sessions a user should then be able to create annotations, 
which can be interacted with (e.g.~edited or deleted) and are tied to the 3D model and the session. 
Beyond this there is a lot of other functionality which should be in place for a complete product, but which will not be a priority 
for this thesis as the virtual reality and gesture recognition aspects are the focus. 

In the final product the application should support a lot more functionality, some of which is described in the next sections.

%To enable the application to a be a VR collaboration tool

\subsection{Application use cases}
\import{}{2-application_functionality.tex}

\section{The gestures}
\label{sec:gesture_design}
\import{../3-design/}{3-gesture-scheme.tex}

\section{Why Leap Motion?}

\section{Why Unity?}


% \section{Functionality limitations}
% \subsection{Handling textual input with gestures}
% \subsection{User gesture calibration}
% \subsection{Saving annotation to a database}
% \subsection{Exposing annotation to web servers}
% \subsection{Annotation time-lines}

% \section{Challenges with VR and GRT} 
% Problems with using VR + e.g Leap over mouse + keyboard + display. E.g:

% \subsection{The "writing issue"}
% Virtual keyboards are bad. 
% Regular keyboards are impractical. 
% See "ideer til masteroppgaven.txt"
% 
% \subsection{Challenges in "designing" gesture schemes}
% People have different preferences. Have intuitive gestures. Have gestures that is not too
% fatiguing. Have gesture with high precision and recall (F-score) (high TP and TN. Low FP, FN).
% Have a system that doesnt mistake one gesture for another.
% 
% \subsubsection{Fixes?}
% User-gesture calibration. 
% 
% \section{Related work}

