% Carsten: 
% This chapter should come much earlier.
% Perhaps not everything, but at least DNV GL should be introduced and their motivations should be stated before even starting with the state-of-the-art reviews. 
% That means that section 5.1 of chapter 5 should be an own chapter between chapters 1 and 2. It would provide the fundamental reasons for making your thesis.
% In that new chapter, you should describe the approval workflow as it is today: papers is sent back and forth, models are made and updated on both sides, 
% annotations are disconnected from designs. And the vision: a share digital design where designer and reviewer can interact, perhaps even meet in a cooperative mode, 
% but without loosing the accountability that comes from today’s paper trail. This story of DNV GL’s vision is important - you must come back to it at the start of chapter 6.
% 
% It should also preferably be longer: more about what DNV GL is, what classification comprises, focus onto ships in this thesis. Where in the long line of 
% classification-related tasks does your thesis fit?
% The rest of 5 is in the right place. However, the current chapter 4 should really come after you have stated your requirement into this chapter.


\section{The core design} 
The core functionality in the application should be to navigate the 3D model and "annotate" it (i.e creating and placing remarks tied to a the model), 
primarily by using the advantages of virtual reality and gesture recognition. 
The users should in later iteration also be able to create "sessions" that enable several users to be virtually present 
in the same instance of the 3D model, and to interact with it using gestures. During these sessions a user should then be able to create annotations, 
which can be interacted with (e.g.~edited or deleted) and are tied to the 3D model and the session. 
Beyond this there is a lot of other functionality which should be in place for a complete product, but which will not be a priority 
for this thesis as the virtual reality and gesture recognition aspects are the focus. 

In the final product the application should support a lot more functionality, some of which is described in the next sections.

%To enable the application to a be a VR collaboration tool

\subsection{Application use cases}
\import{}{2-application_functionality.tex}

\section{The gestures}
\label{sec:gesture_design}
\import{../3-design/}{3-gesture-scheme.tex}

% \section{Functionality limitations}
% \subsection{Handling textual input with gestures}
% \subsection{User gesture calibration}
% \subsection{Saving annotation to a database}
% \subsection{Exposing annotation to web servers}
% \subsection{Annotation time-lines}

% \section{Challenges with VR and GRT} 
% Problems with using VR + e.g Leap over mouse + keyboard + display. E.g:

% \subsection{The "writing issue"}
% Virtual keyboards are bad. 
% Regular keyboards are impractical. 
% See "ideer til masteroppgaven.txt"
% 
% \subsection{Challenges in "designing" gesture schemes}
% People have different preferences. Have intuitive gestures. Have gestures that is not too
% fatiguing. Have gesture with high precision and recall (F-score) (high TP and TN. Low FP, FN).
% Have a system that doesnt mistake one gesture for another.
% 
% \subsubsection{Fixes?}
% User-gesture calibration. 
% 
% \section{Related work}

