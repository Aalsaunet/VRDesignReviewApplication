\section{The core functionality} 
In section~\vref{sec:initial_design} we discussed the fundamental design ideas for a virtual reality design review application. 
To ensure that such an application could change the work flow of DNV GL's design reviews, multiple design aspects should be met and a satisfactory infrastructure would need to be
set up. As this thesis' scope is limited to virtual reality and gesture recognition technology's role in such an application, several of the components necessary for
a satisfactory product will not be implemented to focus more one these aspects. The resulting thesis implementation will thus be more a prototype or proof-of-concept of 
how virtual reality and gesture recognition technology can be used to interact and work with 3D models. 

\subsection{Application use cases}
\import{}{2-application_functionality.tex}

\section{The gestures}
\label{sec:gesture_design}
\import{../3-design/}{3-gesture-scheme.tex}

\section{Technology Choices}
There are several technology choices to make with regards to the implementation of the design review application.
These span from programming framework, language, and what gesture recognition and virtual reality vendors to use, 
and these decisions also must address capability between the different technologies.

% A major decision is what programming framework to use. In the implementation the Unity game engine will be utilized.
\subsection{The Game Engine}
A game engine is a software framework, usually designed for development of video games. 
The core functionality of a game engine typically includes a rendering engine, a physics engine (at least providing collision detection), sound, scripting, 
animation, networking, streaming, memory management, and threading~\citep{Gregory2014}. As game engines are created to enable development of complex 
3D environments and contain many of the facilities necessary for the application use cases outlined above, they provide a good foundation for the implementation.
Even though several game developers develops their own proprietary game engines, which are kept strictly private to the company, there are several commercially available ones 
as well. The biggest of these is the Unity and the Unreal engines, both with broad support from a number of third party vendors. This is a great benefit for the implementation 
as support "straight out of the box" for our choice of virtual reality and gesture recognition technology will ease the development process.

Of these two Unity were chosen as it was requested from DNV GL, as they have more experience with it from other projects.
The Unity engine and its central concepts will be discussed in chapter~\vref{chapter:technical}.


\subsection{Why Leap Motion?}


\subsection{Oculus Rift and HTC Vive}


% \section{Functionality limitations}
% \subsection{Handling textual input with gestures}
% \subsection{User gesture calibration}
% \subsection{Saving annotation to a database}
% \subsection{Exposing annotation to web servers}
% \subsection{Annotation time-lines}

% \section{Challenges with VR and GRT} 
% Problems with using VR + e.g Leap over mouse + keyboard + display. E.g:

% \subsection{The "writing issue"}
% Virtual keyboards are bad. 
% Regular keyboards are impractical. 
% See "ideer til masteroppgaven.txt"
% 
% \subsection{Challenges in "designing" gesture schemes}
% People have different preferences. Have intuitive gestures. Have gestures that is not too
% fatiguing. Have gesture with high precision and recall (F-score) (high TP and TN. Low FP, FN).
% Have a system that doesnt mistake one gesture for another.
% 
% \subsubsection{Fixes?}
% User-gesture calibration. 
% 
% \section{Related work}

