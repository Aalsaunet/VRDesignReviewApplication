The \texttt{CameraRigs} game object is a direct child of the \texttt{MasterController} and holds three different game object, 
which each represents its own camera-setup: \texttt{DesktopCameraRig}, which is meant to be used
without virtual reality, \texttt{OculusCameraRig}, meant to be used with Oculus Rift HMDs, and \texttt{ViveCameraRigs}, meant to be used with the HTC Vive. 
While the desktop rig uses one main camera, the virtual reality rigs (i.e the Oculus and Vive rigs) utilizes two main cameras (one per eye).
These are slightly offset, by about the same length as the real-world distance between two eyes, and rendered separately.
The camera rigs all utilize a separate camera for rendering only annotation objects in the scene, while the main camera(s) (one for desktop, two for VR), 
renders the rest. %This is done by having the annotation on a separate layer (the "AnnotationSphere" level) from the rest of the objects in the scene. 
This is done by:

\begin{enumerate}
	\item Placing the annotation objects in the scene on a different rendering layer than the rest of the objects (called the annotation-layer).
	\item Have the main camera(s) render all layers, except the annotation-layer.
	\item Have the annotation-camera only render objects on the annotation-layer.
	\item "Combine" the output of the two cameras by drawing the output from the annotation camera on top of the output from the main camera. 
		  The main camera thus renders first and the annotation camera second. This is accomplished by giving the main camera \texttt{depth = 0} and the
		  annotation camera \texttt{depth = 1}. 
\end{enumerate}

By rendering these two categories of layers independently we get some flexibility and options with regards to how to present the annotations.
This will be discuss more in depth in the~\vref{sec:annotations} section.

For the application to run successfully one of these rigs should be enabled, while the other two should be disabled.
This can be done by switching between the three rigs in the dropdown-menu named Rig, which is present on the \texttt{CameraRigs} game object itself and
implemented in the \texttt{CameraRigSetup} script. In addition to ensuring that only the correct rig is enabled, 
the \texttt{CameraRigSetup} script also does several other operations. One of these is ensuring that the field of view is set to 60 degrees if the desktop rig 
is selected, as this can wrongfully be set to a HMD's value if a HMD is connected to the computer. When a virtual reality rig is used the field of view is 
set automatically by the HMD software. Another thing done by the script is to decide whether a two dimensional crosshair/cursor should be drawn on the screen space
(in case of the desktop rig), or if a three dimensional crosshair/cursor (i.e the \texttt{GazePointerRing}) should be drawn in the world space. 