This chapter will give a brief introduction to central concepts of the software frameworks outlined in the previous design chapter.
The sections covering the Unity game engine and the Leap Motion Controller are both based on their documentation pages, found at 
https://docs.unity3d.com and https://developer.leapmotion.com.
Note that although these introductions are brief, more detailed information is mentioned as necessary in the implementation chapter.

\section{Unity - The Cross-platform Game Engine}
Unity (formerly Unity3D) is a cross-platform game engine developed by Unity Technologies, and is a popular engine both in a personal- and enterprise settings.
Unity Personal is a free version for individuals and enterprises making less than 100 000\$ a year off content created in Unity, and is the 
edition used in for the implementation phase.
Unity Personal is full-featured and comes with all the necessary subsystems typically required in a game engine, like rendering, physics and scripting.
Unity makes use of either C\# or UnityScript (a dialect of JavaScript) as scripting language, where the former is the preferred by the community and 
exclusively used in the implementation. As its primary software framework Unity used \textit{Mono}, an open source development platform based on the .NET
framework. The main different between the .NET framework and the Mono framework is that mono aims to be platform independent, whereas .NET is Windows only.
The major components of the Mono framework is a C\# compiler, a runtime environment, the .NET class library and a Mono class library.

In Unity, all script which will be using the framework must be derived from the base class \texttt{MonoBehavior}. 
This class contains a lot of key functions in the Unity framework, such as \texttt{Start()} and \texttt{Update()}. 
The following sections will cover some important concepts in Unity.

\import{../6-technical_review/}{unity.tex}

\section{The Leap Motion Controller}
\import{../6-technical_review/}{leap_motion.tex}

\section{Summary}
In this chapter we have reviewed some basic concepts in Unity and in the Leap Motion API, both of which are relevant to understanding the next chapter. 
As will be apparent in the next chapter, these concepts had a big impact on how the design review application was implemented, as 
they are utilized in several key components of the application.
The implementation chapter will also expand upon the concepts reviewed here, and - when relevant - introduce new ones.

%As will be apparent in the next chapter, these concept are very relevant for the implementation of the design review application. 
%The implementation chapter will also expand upon the concept reviewed here where relevant, and put several of the concepts reviewed into a context.
%The next chapter will document this implementation using several of the concept reviewed here, in addition to some more details where necessary.