%Carsten: Add the actual ship model, provided by DNV GL (picture!) to the eval discussion.

In this chapter, we summarize the work presented in this thesis, as well as our findings. After 
this we discuss ideas for future work.


\section{Summary}
In this thesis, we have reviewed how the current state-of-the-art virtual reality- and vision-based gesture recognition technologies can be utilized 
in the design review process of the major international classification society DNV GL. We did so by first reviewing the roles and responsibilities of classification societies 
in general, before discussed DNV GL's current workflow in chapter~\vref{chapter:dnvgl}. In this chapter we also mentioned some initial design ideas for how a full-featured 
design review application might look, and what functionality it could offer to improve DNV GL's design review workflow. This includes
conducting virtual design review meetings, were the designer and reviewer could meet in the 3D model, interact, survey- and annotate it together.
These annotations could be stored in a database - a system keeping track of their history, information and states - and be accessed from multiple platforms - like 
an issue tracker or a digital scrum board. The interaction, surveying, and experience as a whole, could be enhanced by virtual reality technology -
giving the unique sense of scale and a depth that is invaluable for a design - 
and gesture recognition technology - allowing the users to work with the 3D model in new and innovative ways. 

In chapter~\vref{chapter:vr} we reviewed the basics of virtual reality technology and the various challenges of developing a virtual reality system. 
Specifically, we covered the big performance demands a virtual reality system places on a computer, as having a low latency is crucial to avoid user discomfort (such
as virtual reality sickness). To achieve a low enough latency it is required to have a high enough frame rate - i.e the number of frames rendered by the system per second - 
coupled with a high refresh rate - i.e the number of times the display hardware updates its buffers per second. Latency does get better than the weakest of these two, so it is 
important that both the frame- and refresh rate are high enough. Asynchronous reprojection is a method to combat a low frame rate by also inserting "pseudo-frames", 
created by manipulating previously rendered frames, which might give a higher perceived frame rate. A high frame rate without using asynchronous reprojection is still 
recommended. A sufficiently high display resolution and pixel density are also important perform in virtual reality system, both to avoid the screen-door effect as well
as positional judder. Virtual reality also impacts how VR-application should be design and implemented. Failing to meet these demands can increase the 
risk of virtual reality sickness, a condition similar to motion sickness. There are several factors that impact a virtual reality-user's susceptibility for virtual 
reality sickness, which can be dividing into those caused by individual differences and those caused by the application or virtual reality system. 

In chapter~\vref{chapter:grt} we reviewed the basics of gesture recognition technology and discussed the two primary categories of gesture recognition devices: Vision-based
and contact-based. We compared the two and concluded that the design review application would target a vision-based gesture recognition system, as these have several important
advantages over contact-based, such as having lower health risk associated with them and a higher user friendliness. We also discussed the three primary vision-based 
technologies - stereoscopic vision, structured light and time of flight - and concluded that we would target a stereoscopic vision system, as these are 
arguably the most promising for use in an office environment. After this we discussed the three stages of gesture recognition - detection, tracking and recognition - 
and several approaches to each of these phases. In the chapters conclussion we conclude to use the Leap Motion Controller as our gesture recognition system, 
as it is both a stereoscopic vision system - and thus also vision-based - and also because it has a well documented high-level API and an attractive size and price.

% In chapter~\vref{chapter:grt} 







%%%%%%%%%%%%%%%%%%%%%%%
This thesis reviews how the current state-of-the-art virtual reality- and vision-based gesture recognition technologies can be utilized in a professional capacity, 
and more specifically, to the design reviews of complex 3D models. 
Instead of printing the model to paper, drawing on the paper, scanning it and sending it by email - as isn't an uncommon workflow today, 
the designer and reviewer could have a virtual design review meeting, were they could meet in the 3D model, interact, survey- and annotate it together. 
They could manipulate the model, and change how the annotations appear in the models.
These annotations could be stored in a database - a system keeping track of their history, information and states - and be accessed from multiple platforms - like 
an issue tracker or a digital scrum board. The interaction, surveying, and experience as a whole, could be enhanced by virtual reality technology -
giving the unique sense of scale and a depth that is invaluable for a design - 
and gesture recognition technology - allowing the users to work with the 3D model in new and innovative ways. 

This is the vision of DNV GL, who regularly conducts such design reviews. % Hmm, ta bort
To address this, and evaluate this vision's feasibility, this thesis implements such a design review application, and 
reviews its requirements and design, documents its implementation details and evaluates its performance by conducting user test sessions.
In the prototype application, developed as part of this thesis, the user is able to navigate 3D models, annotate them - by either creating 
annotation orbs, which are physical object in the model, or by attaching an annotation directly to an object - and perform various other actions. 
This can be done in a conventional manner, i.e by mouse, keyboard and a display, by only using gestures (hand movements) and a virtual reality headset, or 
a combination of the two.

To ensure that the application is developed using state-of-the-art technology, and that it addresses the challenges such technology presents, 
the field of virtual reality is reviewed and discussed. In addition to this, the gesture recognition field is also reviewed, as this alternate 
input method can have the potential to increase the usability of 3D-based applications - such as the design review application -, especially when coupled with virtual reality.
Several software frameworks and devices that enable this technology are also discussed in this thesis, as we wanted to quickly prototype the 
application using these state-of-the-art tools.
We chose to use readily available virtual reality tools, such as the virtual reality headsets Oculus Rift CV1 and HTC Vive, coupled with
the Leap Motion Controller, a stereoscopic vision-based gesture recognition device. The software was developed using the Unity 
framework - a popular game engine - because of its capability and as it allowed us to interface with other projects in DNV GL


\section{Future Work}







% This essay has given a brief summary of the virtual reality design review application that is going to be implemented for DNV GL 
% as part of the master's thesis, and how virtual reality- and gesture recognition technology can be utilized to potentially improve the 
% human-computer interaction experience beyond that of more conventional interaction methods. 
% 
% Gesture recognition technology is often divided into the categories of vision-based and contact-based, where the former usually is the 
% preferred one because of user-friendliness and the health concerns associated with the latter. Vision-based gesture recognition devices usually 
% utilize either stereoscopic vision-, structured light pattern- or time of flight techniques, where stereoscopic vision-based devices have proved the most promising. 
% One device of this kind is the Leap Motion Controller, which consists of two stereoscopic cameras and three infrared LEDs and periodically captures 
% grayscale stereo images which are sent to the tracking software, where 3D representations are constructed. 
% 
% The master's thesis aims to evaluate the performance and user experience of utilizing a Leap Motion Controller in combination with the Oculus 
% Rift and HTC Vive virtual reality headsets during a virtual design review in a complex 3D model. The final application should thus be primarily 
% focused on utilizing the most intuitive ways of interacting with complex 3D models in a collaborative virtual reality setting.     	                                     


% Through previous research endeavours vision-based gesture recognition devices has proved to be the most promising ones, 
% and it has already seen some moderate commercial success through devices such as the XBox Kinect and the Leap Motion Controller. 
% However, many previous research articles states that the techniques implemented for hand gesture recognition often are sensitive to poor resolution, 
% frame rate, drastic illumination conditions, changing weather conditions and occlusions among other prevalent problems in the hand gesture recognition 
% systems~\citep{Rautaray2015}.

% Despite these reported shortcomings, the upcoming master's thesis will explore areas where gesture recognition technology might perform in a sufficient manner, 
% and also explore how a gesture recognition system can be used in combination with more traditional input-methods, such as a mouse, keyboard, gamepad or joystick. 
% The final application implementation should thus be primarily focused on utilizing the most intuitive ways of interacting with complex 3D models in a virtual reality setting.   

% the use cases for these are still somewhat limited and has often been reported to not have a reliable enough tracking and recognition~\citep{Guna2014}. 