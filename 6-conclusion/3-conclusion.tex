This essay has given a brief summary of the virtual reality design review application that is going to be implemented for DNV GL as part of the master's thesis, and how virtual reality- and gesture recognition technology can be utilized to potentially improve the human-computer interaction experience beyond that of more conventional interaction methods. 

Gesture recognition technology is often divided into the categories of vision-based and contact-based, where the former usually is the preferred one because of user-friendliness and the health concerns associated with the latter. Vision-based gesture recognition devices usually utilize either stereoscopic vision-, structured light pattern- or time of flight techniques, where stereoscopic vision-based devices have proved the most promising. One device of this kind is the Leap Motion Controller, which consists of two stereoscopic cameras and three infrared LEDs and periodically captures grayscale stereo images which are sent to the tracking software, where 3D representations are constructed. 

The master's thesis aims to evaluate the performance and user experience of utilizing a Leap Motion Controller in combination with the Oculus Rift and HTC Vive virtual reality headsets during a virtual design review in a complex 3D model. The final application should thus be primarily focused on utilizing the most intuitive ways of interacting with complex 3D models in a collaborative virtual reality setting.     	                                     

%Through previous research endeavours vision-based gesture recognition devices has proved to be the most promising ones, and it has already seen some moderate commercial success through devices such as the XBox Kinect and the Leap Motion Controller. However, many previous research articles states that the techniques implemented for hand gesture recognition often are sensitive to poor resolution, frame rate, drastic illumination conditions, changing weather conditions and occlusions among other prevalent problems in the hand gesture recognition systems~\citep{Rautaray2015}.

%Despite these reported shortcomings, the upcoming master's thesis will explore areas where gesture recognition technology might perform in a sufficient manner, and also explore how a gesture recognition system can be used in combination with more traditional input-methods, such as a mouse, keyboard, gamepad or joystick. The final application implementation should thus be primarily focused on utilizing the most intuitive ways of interacting with complex 3D models in a virtual reality setting.   

%the use cases for these are still somewhat limited and has often been reported to not have a reliable enough tracking and recognition~\citep{Guna2014}. 